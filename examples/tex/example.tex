\documentclass{article}
\usepackage[a4paper, margin=1in]{geometry}
\usepackage{fontspec}


\begin{document}

\fontsize{25pt}{30pt}
\fontspec{Slabikar.otf}

\noindent A--- písmeno pro tebe,\\
anděl letí do nebe,\\
Andělka mu štěstí přála\\
a šátečkem zamávala,\\
aby neulét\\
a vrátil se zpět.\\

\noindent Digits: 1234567890\\
Punctuation: !?,.:;'" a-a a--a a---a\\
Other: +-×*=/\textbackslash <>@()[]\{\}\%\&\$\#\^{}|\~{}\_‰\\
"x" 'x' „x“\\
aáäbcčdďeéěfghiíjklĺľmnňoóôpqrřŕsštťuúůvwxyýzž\\
AÁÄBCČDĎEÉĚFGHIÍJKLĹĽMNŇOÓÔPQRŘŔSŠTŤUÚŮVWXYÝZŽ\\
As Ás Äs Bs Cs Čs Ds Ďs Es És Ěs Fs Gs Hs Is Ís Js Ks Ls Ĺs Ľs Ms Ns Ňs Os Ós Ôs Ps Qs Rs Řs Ŕs Ss Šs Ts Ťs Us Ús Ůs Vs Ws Xs Ys Ýs Zs Žs\\


\def\TeX{T\kern-.28em\lower.7ex\hbox{E}\kern-.17emX}
\righthyphenmin=2  \emergencystretch=2em
\hbadness=2900

Toto písmo je potřeba brát spíš jako příklad, co všechno \TeX{} dovede.
Nepředpokládám velké nasazení tohoto písma pro sazbu příštích
slabikářů. Dokonce takovou věc ani nedoporučuji.

Všechny ukázky ve slabikáři a v~písankách, které jsem měl možnost vidět,
jsou psány lidskou rukou a ne strojem. Samozřejmě smekám před kaligrafem,
který ty ukázky vytvořil. Člověk má na první pohled dojem, že to je
„jak když tiskne“. Fušoval jsem také do kaligrafického řemesla, a proto
dobře vím, že pokud písmo neobsahuje žádné ozdobné prvky, musí to napsat
skutečně profesionál. Každá chybička, která by se třeba skryla za ozdobným
prvkem, je totiž vidět.

Důležité ale je, že písmo v~dnešním slabikáři bylo skutečně napsáno jen
„jak“ když tiskne a nikoli tištěno doopravdy. Písmu tak neschází
lidský rozměr, který ten prvňák podvědomě z~toho písma asi cítí. Kdyby se
pro sazbu ukázek použil stroj (třebaže~\TeX), písmo by tento rozměr
ztratilo. Takové písmo by bylo chladné, stále stejné, bez výrazu,
tedy vlastně mrtvé. Nepřeji prvňákům, aby se někdy v~budoucnu
s~takovým chladným písmem setkali.

\medskip
\hfill Petr Olšák


\end{document}
